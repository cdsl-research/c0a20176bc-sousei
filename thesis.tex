% thesis.tex -- 論文の書き方参考例
%
% (注意)氏名、学籍番号等を変更すること。
%
% (LaTeXの実行法) platex thesis.tex
%
%	このファイル内にある '% 'はコメント。
%
\documentclass[a4j,11pt]{jreport}
\usepackage{ascmac}
\usepackage{amsmath}
\usepackage{float}
\usepackage{url}
\usepackage[dvipdfmx]{graphicx}
\usepackage{styles/thesis-master}
\usepackage{styles/mylatex}
\pagestyle{plain}

% 所属研究科、専攻
\courseofmastercs
% 研究テーマ
\title{論文タイトル}
% 氏名
\author{小~~山~~智~~之}
% 学籍番号
\id{G~~2~~1~~2~~1~~0~~2~~4}
% 指導教員
\teacher{串~田~~高~幸}
% 提出日
\date{2023}{1}{xx}
% 提出年度
\schoolyear{2022}

% 研究室名(カバー用)
\clab{クラウド・分散システム}
% 学籍番号(カバー用)
\cid{G2121024}

%% 英語要旨
\etitle{Title in English}
\eauthor{Tomoyuki~~Koyama}
\eteacher{Takayuki~~Kushida}

\begin{document}
%\makecover		% カバー(提出電子化に伴い削除)
\maketitle		% 表紙

% 修士論文和文要旨
\jabst{
  \input{00jabstract.tex}	% 00jabstract.texを読み込む
}
\makejabstract

% 修士論文英文概要
\eabst{
  \input{00eabstract.tex}
}
\makeeabstract

% 目次
\pagenumbering{roman}
\setcounter{page}{1}
\thispagestyle{plain}
\tableofcontents	% 目次
% \listoffigures		% 図目次
% \listoftables		% 表目次
% \listofprograms		% プログラム目次

% 論文本文
% 論文本文は章や節単位で,いくつかのファイルに分割したほうが編集しやすい.
% \input{hogehoge}	% hogehoge.texを読み込む.
% 章構成は適宜変更すること.
\pagenumbering{arabic}
\chapter{はじめに}

\section{背景}

サンプル\cite{teu}


\section{課題}

サンプル


\section{目的}

サンプル

\chapter{関連研究}

サンプル
\chapter{提案方式}

\section{提案方式}

サンプル

\section{ユースケースシナリオ}

サンプル
\chapter{設計と実装}

\section{全体構成}

サンプル

\section{ソフトウェアの構成}

サンプル

\section{実装}

サンプル

\chapter{実験環境}

\section{実験環境}

サンプル

\section{検証方法}

サンプル

\section{実験手順}

サンプル
\chapter{実験結果と分析}

\section{実験結果}

サンプル

\section{分析}

サンプル

\chapter{議論}

サンプル
\input{08conclusion}

\bibliographystyle{junsrt}
\bibliography{reference} % 参考文献
\input{20publications} % 業績
% appendix.tex -- 付録
\appendix
% \chapter{ソースコード}

% リストファイルを\pgref{pg:CONTENT}に示す.

% \lstinputlisting[caption=CONTENT,
% label=pg:CONTENT]{00CONTENT}
	% 付録



\end{document}
